\begin{table}[h!t]
	\begin{tabular}{lp{11cm}}
	Vértice   & Es el punto donde concurren las dos semirrectas que conforman un ángulo.\\

	Arista    & Es, en geometría, el segmento de recta donde intersectan dos planos. Por extensión también se conoce con este
			nombre al segmento común que tienen dos caras vecinas de un poliedro.\\

	Poliedro  & Un poliedro es, un cuerpo geométrico cuyas caras son planas y encierran un volumen finito. La palabra poliedro
			viene del griego clásico \textgreek{polyedron} \emph{(polyedron)}, de la raíz \textgreek{poolys} \emph{(polys})
			\jcq{muchas} y de \textgreek{edra} \emph{(edra)} \jcq{base}, \jcq{asiento}, \jcq{cara}.\\

	Tetraedro & Un tetraedro es un poliedro de cuatro caras. Con este número de caras ha de ser un
			poliedro convexo, y sus caras triangulares, encontrándose tres de ellas en cada vértice. Si las
			cuatro caras del tetraedro son triángulos equiláteros, iguales entre sí, el tetraedro se denomina
			regular.\\

	Píxel     & Un píxel (acrónimo del inglés picture element, \jcq{elemento de imagen}) es la menor unidad homogénea en color
			que forma parte de una imagen digital.\\

	Voxel     & El vóxel (del inglés volumetric pixel) es una unidad cúbica que compone un objeto
			tridimensional. Constituye la unidad mínima procesable de un espacio tridimensional y es, por
			tanto, el equivalente del píxel en un objeto 3D.\\

	Dataset   & Un \emph{dataset} es un conjunto de datos de entrada, para esta
			implementación, un \emph{dataset} representa un conjunto de imágenes.\\

	Isovalor	& Es un valor constante dentro de el rango definido por la profundidad de 					color de las imágenes del \emph{dataset}\\

	Isosuperficie	& Es aquella superficie que representa los puntos de un valor constante o \emph{isovalor} (presion, temperatura, etc.) dentro de un volumen en un espacio. En otras palabras, dado un espacio escalar $f(x,y,z)$ en una region $\mathbb{R}$, una \emph{isosuperficie} es una superficie determinada por el \emph{isovalor} $\alpha$, en la que todos sus puntos cumplen: $f(x,y,z) = \alpha$.\\

	\end{tabular}
\end{table}
