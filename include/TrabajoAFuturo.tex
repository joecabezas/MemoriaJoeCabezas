\chapter{Trabajo A Futuro}
\label{ch:trabajoafuturo}

En esta investigación se presentó un flujo de trabajo en la sección \ref{ch:propuesta:sec:flujoDeTrabajo}, y se hizo especial énfasis en el paso de extracción de la superficie explicado en la sección \ref{ch:propuesta:sec:extraccionDeLaSuperficie}. Este proceso puede ser mejorado considerando los siguientes puntos.

\begin{itemize}
	\item La posición de los vértices obtenidos en la malla de superficie generada por el algoritmo, es calculada usando interpolación lineal, la cual no necesariamente es parecida al valor real de la superficie estudiada.

	\item La elección del isovalor es dentro de una escala porcentual, la escala no garantiza que superficies obtenibles con más de cero triángulos estén distribuidas dentro de este rango, es posible aún que, dependiendo del \emph{dataset}, el conjunto de superficies con más de cero triángulos se encuentre dentro de un solo valor porcentual, una posible solución, es analizar y determinar peviamente, usando búsqueda binaria, y con una cantidad limitada de iteraciones, aquellos valores porcentuales del isovalor que generen superficies vacías y con ello ampliar la escala para que un valor porcentual de cero del isovalor, signifique que es el valor mas bajo donde se existe una superficie no vacía, y un valor de cien por ciento, el valor mas álto encontrado.

	\item El rendimiento puede mejorarse usando paralelismo, debido a que esta implementación hace un recorrido de los cubos sin ocupar información de los cubos vecinos, se puede dividir el \emph{dataset} para que distintas unidades de procesamiento analicen la totalidad de los cubos del \emph{dataset}.

	\item El algoritmo usa una resolución fija para dividir el espacio, y está determinado por la resolución y cantidad de imágenes del \emph{dataset}, lo cual puede generar más triángulos de los necesarios, por ejemplo para representar una superficie plana, esto puede mejorarse usando técnicas adaptativas\cite{Shu95adaptivemarching} que reduzcan la cantidad de triángulos que representan la superficie adaptando el tamaño de los triángulos según la forma de la superficie.

	\item La implementación ocupa el algoritmo expuesto por Lorensen y Cline\cite{Lorensen87marchingcubes}, el cual tiene ciertos problemas asociados a la topología de las superficies obtenidas explicadas en la sección \ref{subsec:marchingCubes:consecuencias}, es posible usar una versión alternativa del algoritmo que disminuya estos problemas\cite{Bloomenthal88polygonizationof}\cite{Chernyaev95marchingcubes}\cite{BAPayne90surfacemapping}\cite{Shu95adaptivemarching}.
\end{itemize}