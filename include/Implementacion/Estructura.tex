\section{Estructura}
\label{ch:implementacion:sec:estructura}

Para realizar la implementacion del algoritmo se debe primero proponer una convención para enumerar los vertices y aristas de un cubo, para esta implementacion se usará la convención señalada anteriormente en la figura \ref{f:estadoDelArte:convention}, luego de definida la enumeración de cada vértice y arista, es necesario poder representar cada uno de los 256 casos con un identificador único denominado \emph{cubeIndex}, luego hay que determinar que aristas son cortadas por la superficie que atraviesa cada caso en particular, para ello se necesita de una tabla que asocie un \emph{cubeIndex} con las aristas que serán atravesadas, esta tabla se denomina \emph{edgeTable}, una vez que se conocen aquellas aristas que son atravesadas, el siguiente paso es crear una malla de triángulos que formen esta superficie, la tabla \emph{triTable} asocia cada caso con una lista de triángulos ordenados, que generan la superficie buscada. Todas estas estructuras de datos serán explicadas a continuación.