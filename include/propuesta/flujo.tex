\section{Flujo de trabajo}
\label{ch:propuesta:sec:flujoDeTrabajo}

Actualmente existen flujos de trabajo para la extracción de supeficies \cite{Ruprecht94ascheme}\cite{Dietrich_marchingcubes}, los cuales intentan generar superficies sin los principales problemas topológicos explicados en \ref{subsec:marchingCubes:consecuencias}. El flujo de trabajo propuesto en esta investigación se muestra en la fugura \ref{f:estadoDelArte:flujoDeTrabajo}

\begin{figure}[htb]
\centering
	\includegraphics[width=0.5\textwidth]{images/misc/workflow.pdf}
\caption{Flujo de trabajo propuesto}
\label{f:estadoDelArte:flujoDeTrabajo}
\end{figure}

Cada uno de estos pasos serán explicados a continuación.

\subsection{Extracción de datos}
\label{ch:propuesta:sec:extraccionDeDatos}

%incluir que es un MRI
In medical image analysis one often needs to visualize volumetric data sets, such as the
measurements coming out of a MRI scanner. MRI images can be thought of as 3D images
where each pixel (or voxel - volume element) represent a measured quantity in a small
volume of space at some location (x,y,z). One possible way of visualizing MRI images is
by isosurface rendering using the marching cubes algorithm. An isosurface is dened as
the set of points with the same voxel value. Your task would be to implement such an
isosurface renderer.