\chapter{Introduction}
\label{ch:introduction}

\section{Mallas Geométricas}
\label{sec:mallasGeometricas}
Las mallas de superficie son una herramienta fundamental en la ciencia e ingeniería, son
un conjunto de polígonos (triángulos, cuadriláteros, etc.), que conforman una superficie en un
espacio definido, en el que la intersección de dos elementos de una malla, puede ser el conjunto
vacío, un vértice, una arista o una cara en común. Las mallas tienen asociadas un conjunto de
elementos topológicos tales como: vértices, aristas, y caras poligonales.
Regularmente, las mallas geométricas se usan para describir, dependiendo su dimensión,
una región en un plano, o un volumen en un espacio. Realizar esta discretización consiste en
aproximar el dominio a simular dividiéndolo en elementos geométricos mas sencillos,
comúnmente triángulos, de tal manera que la intersección de dos elementos, sean un vértice, una
arista, o el conjunto vacío.

Existen casos triviales como la discretización de un dominio rectangular, cuya solución es
directa, dividiendo el dominio en un arreglo uniforme de triángulos, pero cuando se desea
discretizar un elemento de geometría irregular, la solución no es tan trivial, y se necesita de un
método de generación de mallas más elaborado.

Sin embargo, para muchos problemas de ingeniería, las mallas generadas por la
triangulación de Delaunay\cite{Delaunay1934}, no son suficientes para resolver un problema o su análisis numérico. En algunos casos es posible que sea necesario agregar nuevos puntos con el fin de mejorar la
calidad de la malla, a este proceso de mejoramiento, se le conoce como Refinamiento.

En tres dimensiones, en particular, las mallas geométricas son una representación de un
objeto físico, y para efectos de su visualización, los elementos internos de una malla pueden ser
omitidos, y con ello, sólo se utilizan mallas que describan la superficie del objeto.

\section{Usos de las Mallas Geométricas}
\label{sec:usosDeLasMallasGeometricas}
Las mallas geométricas tienen diferentes usos en la práctica, desde simulaciones físicas
tales como, estudio de fuerzas, interacción de objetos, simulación de fluidos, etc. También tienen
usos gráficos, como para hacer visualizaciones de cuerpos en tres dimensiones, visualización de
funciones matemáticas, e incluso usos artísticos como gráficos realistas, películas, etc.


\section{Consideraciones Físicas}
\label{sec:consideracionesFisicas}
En la realidad, vivimos inmersos en un mundo que podemos describirlo en tres
dimensiones, y con ello, cualquier partícula (objetos, personas, etc.) tienen una posición
específica dentro de un sistema de referencia continuo, es decir, que no existe una distancia
mínima para el desplazamiento.

Por ejemplo, usando un acercamiento a la definición de límites: si un objeto en un espacio
continuo se desplaza una distancia  d  positiva  ($d > 0$) , siempre existirá una partícula que
pueda desplazarse una distancia menor, sin importar lo pequeño que sea el valor de $d$.

Es esta continuidad del espacio del mundo real, lo que dificulta la tarea de representar
cualquier objeto en una simulación virtual.
En un espacio en tres dimensiones simulado computacionalmente, debido principalmente
a los límites de memoria de los sistemas actuales, no pueden simular un espacio continuo, sino
que todo elemento dentro de un espacio virtual debe estar inmerso en un espacio discreto, es
decir, que existe una cantidad mínima para el desplazamiento.

Los objetos en el mundo real están compuestos de materia, la que está compuesta por
átomos, la unión de ellos, crean objetos sólidos, con un nivel de detalle tan fino como atómico,
debido a la continuidad del espacio real.

Los objetos en un mundo virtual, no están (ni pueden aún), ser descritos por unidades
atómicas, ya que supondría una cantidad enorme y muy costosa de memoria para almacenar cada
átomo, y por ello, los objetos son representados por una cantidad discreta de triángulos que en
conjunto forman una superficie la cual describe la forma visible del objeto representado.

Debido a la carencia de continuidad del espacio virtual, es que los objetos representados
ahí, no son más que simples discretizaciones de los objetos reales, y en consecuencia, sin las
propiedades físicas que presentan los objetos reales, como la densidad, roce, etc.
Por ejemplo, en el mundo real, podemos pulir una superficie de manera que sea
perfectamente lisa y no presente roce, pero por muy lisa que parezca, debido a la continuidad,
podemos acercarnos de manera infinita a la superficie y notar que en realidad, aún tiene
imperfecciones que son mas pequeñas que lo mínimo visible por el ojo humano, y con ello,
eventualmente la superficie presenta roce de todas maneras por estas imperfecciones
microscópicas. Sin embargo, al representar esta superficie en el mundo virtual, simplemente
constaría de un plano perfecto en tres dimensiones, el cual, es perfectamente liso, y sin la
cantidad infinita de detalles que presenta la superficie en la realidad.

\section{Objetos en tres dimensiones}
\label{sec:objectosEnTresDimensiones}
Los objetos en tres dimensiones, como anteriormente se ejemplificó, son representados
principalmente por planos triangulares que en conjunto forman una superficie que describe la
forma externa y visible del objeto. El proceso completo para obtener un objeto en tres dimensiones, depende de la tecnología usada y el propósito por el cual se genera un modelo en
dos o tres dimensiones.

Uno de los primeros algoritmos de generación de mallas, fue el propuesto por el
matemático Ruso Boris Nikolaevich Delanuay\cite{Delaunay1934}. En honor a su autor, este algoritmo de
triangulación se denomina como el \jcq{Algoritmo de Triangulación de Delanuay}, el cual a partir de
una colección de puntos en un espacio, genera una malla geométrica formada por triángulos que
sean lo más equiláteros posible.

Para extraer la información de la forma de un cuerpo, existen tecnologías capaces de
extraer la superficie de éstos, tales como \emph{software} capaces de generar un modelo en tres
dimensiones a partir de un arreglo de fotografías que rodean al objeto, también existe la
tecnología de usar un arreglo de cámaras que capturan, en tiempo real, una nube de puntos
infrarrojos que son emitidos desde el aparato, y con ello, crear un modelo en tres dimensiones. En
el ámbito médico, también existe tecnología para poder extraer información del interior del
cuerpo humano, usando equipos de imagenología de resonancia magnética, la cual es una técnica
no invasiva que utiliza el fenómeno de la resonancia magnética para obtener información de la
estructura y composición del cuerpo a analizar, luego, la información obtenida es procesada a
través de un \emph{software} que genera la malla geométrica del área de interés por el científico.

\section{Objetivos}
\label{sec:objetivos}
En esta investigación, se propondrá un flujo de trabajo que permita la extracción de una superficie dada una colección de puntos en un espacio de tres dimensiones. Se hará enfasis en los primeros pasos del flujo propuesto, para ello se estudiará e implementará el algoritmo de \emph{Marching Cubes} para hacer la extracción de una malla de superficie que deberá ser refinada en los pasos posteriores del flujo propuesto.

Además, se construirá una herramienta de visualización y navegación que ayudará a la correcta elección del valor mínimo para poder extraer una superficie con un nivel de detalle controlable por el investigador.