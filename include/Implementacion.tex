\chapter{Implementación}
\label{ch:implementacion}

\section{Estructura}
\label{ch:implementacion:sec:estructura}

Para realizar la implementacion del algoritmo se debe primero proponer una convención para enumerar los vertices y aristas de un cubo, para esta implementacion se usara la convencion señalada anteriormente en la figura \ref{f:estadoDelArte:convention}, luego de definida la enumeración de cada vértice y arista, es necesario poder representar cada uno de los 256 casos con un identificador único denominado \emph{cubeIndex}, luego hay que determinar que aristas con cortadas por la superficie que atraviesa cada caso en particular, para ello se necesita de una tabla que asocie un \emph{cubeIndex} con las aristas que seran atravesadas, esta tabla se denomina \emph{edgeTable}, una vez que se conocen aquellas aristas que son atravesadas, el siguiente paso es crear una malla de triángulos que formen esta superficie, la tabla \emph{triTable} asocia cada caso con una lista de triángulos, en orden, que generan la superficie buscada, todas estas estructuras de datos seran explicadas a continuación.

\section{Estructuras de datos}
\label{ch:implementacion:sec:estructurasDeDatos}

\subsection{CubeIndex}
\label{ch:implementacion:sec:CubeIndex}

Como se menciono en \ref{subsec:marchingCubes:consideracionesGeometricas}, existen 256 formas posibles de atravesar un cubo con una superficie continua, que es lo mismo decir, existen 256 combinaciones posibles dados 8 vertices que pueden estar en 2 estados distintos (indicando que estan dentro de una superficie cerrada o fuera de ésta)

Para poder identificar cada uno de estos 256 casos, se enumeran del 1 al 256 usando un arreglo de 8 bits, basandose en el estado de cada vertice usando la convención, por ejemplo, un cubo que tiene todos sus vértices marcados como fuera de la superficie (externos), tiene todos sus bits en cero, por lo tanto, el índice, desde ahora \emph{cubeIndex}, de este cubo es \hbox{$0_{10} = 0000 \; 0000_{2}$} (cero), de la misma manera, usando la convención, si un cubo sólo tiene el tercer vertice (vertice 2), dentro de la superficie, entonces, tiene su tercer bit en 1, y el resto en cero, luego el \emph{cubeIndex} del cubo es $4_{10} = 0000 \; 0100_{2}$, si un cubo es atravesado por la mitad, dejando a la mitad de abajo dentro de la superficie, tiene los vértices 4, 5, 6 y 7 marcados como internos, por lo que los bits 4, 5, 6 y 7 (los ultimos 4), valen 1, por lo tanto, el \emph{cubeIndex} es $240_{10} = 1111 \; 0000_{2}$.

En general, el \emph{cubeIndex} se determina como se muestra en la figura \ref{f:ch:implementacion:sec:CubeIndex:cubeindex:cubeindex}.

\begin{figure}[hbt]
	\makebox[\textwidth]{\framebox[0.3\textwidth]{\rule{0pt}{0.2\textwidth}}}
	\caption{Five by Five in Centimetres.}
	\label{f:ch:implementacion:sec:CubeIndex:cubeindex:cubeindex}
\end{figure}

\subsection{edgeTable}
\label{ch:implementacion:sec:edgeTable}

Luego de establecer como identificar un cubo, es necesario poder conocer que aristas serian intersectadas por la superficie, para poder determinar un punto sobre estas aristas por las cuales se sostendrá un triángulo. Por ejemplo, si el vértice $0$, es el unico vértice que queda dentro de la superficie, usando la convención, se puede asumir las aristas $0$, $3$ y $8$ seran aquellas las cuales la superficie intersectara a cubo.

Es por esto que se necesita una forma de relacionar un \emph{cubeIndex} con las aristas que serán atravesadas por la superficie.

Usando la misma estrategia que con el \emph{cubeindex} existen 4096 $(2^{12})$ combinaciones posibles de tomar 12 aristas que pueden intersectar a la superficie o no, por esto, cada caso, será identificado con un numero de 12 bits, en el cual, cada bit representa a una arista, usando la convención como se muestra en la figura \ref{f:ch:implementacion:sec:CubeIndex:edgeTable:edge_convention}.

\begin{figure}[hbt]
	\makebox[\textwidth]{\framebox[0.4\textwidth]{\rule{0pt}{0.3\textwidth}}}
	\caption{Convención para enumerar las 4096 combinaciones posibles de aristas}
	\label{f:ch:implementacion:sec:CubeIndex:edgeTable:edge_convention}
\end{figure}

La \emph{edgeTable} es un \emph{array} (arreglo) diseñado para asociar un \emph{cubeindex} con quellas aristas que son intersectadas. este \emph{array} consta de 256 numeros (uno por cada caso o cada \emph{cubeindex}) de 12 bits, un bit para cada una de las 12 aristas del cubo en cuestión por las cuales pasa la superficie.

Para entender mejor, se supone el ejemplo de la figura \ref{f:ch:implementacion:sec:CubeIndex:edgeTable:example}.

\begin{figure}[hbt]
	\makebox[\textwidth]{\framebox[0.3\textwidth]{\rule{0pt}{0.2\textwidth}}}
	\caption{Five by Five in Centimetres.}
	\label{f:ch:implementacion:sec:CubeIndex:edgeTable:example}
\end{figure}

En este ejemplo, sólamente el vértice $0$, ha sido marcado como interno, luego, el \emph{cubeIndex} es $1_{10} = 0000 \; 0001_{2}$, dado esto, las aristas $0$, $3$ y $8$ serán eventualmente atravesadas por la superficie, por lo tanto:

\begin{quote}
	edgeTable[1] = 0x109
\end{quote}

Lo cual tiene un valor equivalente a: $109_{16} = 265_{10} = 0001 \; 0000 \; 1001_{2}$, lo que indica, viendo la notacion binaria, que las aristas $0$, $3$ y $8$ son las que serán atravesadas.

Otro ejemplo, se supone el caso de la figura \ref{f:ch:implementacion:sec:CubeIndex:edgeTable:example}

\begin{figure}[hbt]
	\makebox[\textwidth]{\framebox[0.3\textwidth]{\rule{0pt}{0.2\textwidth}}}
	\caption{Five by Five in Centimetres.}
	\label{f:ch:implementacion:sec:CubeIndex:edgeTable:example}
\end{figure}

En este caso, los vértices $5$ y $6$ han sido marcados como internos, luego, el \emph{cubeIndex} debe ser: $96_{10} = 0110 \; 0000_{2}$, por lo tanto la superficie deberia atravesar las aristas $4$, $9$, $6$ y $10$, para que asi, una superficie deje a estos vértices separados del resto.

Para saber que aristas finalmente son atravesadas, se debe consultar a la tabla \emph{edgeTable}, usando como índice, el \emph{cubeIndex} calculado anteriormente.

\begin{quote}
	edgeTable[96] = 0x650
\end{quote}

El valor entregado por la \emph{edgeTable} para el \emph{cubeIndex} $96$, es $0x650$ (hexadecimal), lo que es equivalente a:

\begin{quote}
	$650_{16} = 1616_{10} = 0110 \; 0101 \; 0000_{2}$
\end{quote}

Según la notación binaria, se puede ver que el valor entregado por la tabla \emph{edgeTable}, indica que los vertices $4$, $6$, $9$ y $10$ son aquellos que son atravesados por la superficie, de forma que los vertices $5$ y $6$ queden separados del resto.

\subsection{triTable}
\label{ch:implementacion:sec:triTable}

Una vez que se conocen aquellas aristas que seran atravesadas, es momento de crear la superficie interna que atraviese estas aristas y separe los vertices marcados de los demás. Para ello, es necesario poder generar aquellos triángulos que formen esta superficie.

la tabla \emph{triTable} es una tabla de 256 \emph{arrays} (arreglos) de 16 numeros cada uno, hay un arreglo por cada uno de los 256 casos descritos en \ref{ch:implementacion:sec:CubeIndex}, en cada caso, es necesario generar una serie de triángulos que describirán la superficie, y de todos los casos, como máximo se necesitan 5 triángulos para los casos mas complejos.

Analizando un caso simple, un cubo que solamente tiene un vértice marcado, por lo tanto se necesita tan sólo un triángulo que se sostenga de las aristas que convergen en ese vertice, para asi crear una superficie que atraviese el cubo dejando a ese vértice separado del resto de los vértices del cubo, es por esto, que el segundo elemento (\emph{cubeIndex} = 1) de la tabla \emph{triTable} tiene un valor:

\begin{quote}
	$triTable[1] = \{0, 8, 3, -1, -1, -1, -1, -1, -1, -1, -1, -1, -1, -1, -1, -1\}$
\end{quote}

Esto indica que el primer triángulo, es aquel que se sostiene (o corta) de las aristas: $0$, $8$ y $3$.

El resto de los $13$ números que quedan tienen un valor de $-1$, lo cual permite a cualquier implementacion del algoritmo, iterar en los numeros como tripletas, y detenerse cuando el primer número tenga un valor de $-1$.

De esta manera, cuando un cubo no tiene ninguno de sus vértices marcados (\emph{cubeIndex} = 0), o tiene todos sus vértices marcados (\emph{cubeIndex} = 255), los valores de la triTable para ambos casos son:

\begin{quote}
	$triTable[0]	= \{-1, -1, -1, -1, -1, -1, -1, -1, -1, -1, -1, -1, -1, -1, -1, -1\}$\\
	$triTable[255]	= \{-1, -1, -1, -1, -1, -1, -1, -1, -1, -1, -1, -1, -1, -1, -1, -1\}$
\end{quote}

Esto indica que para ambos casos, no se crea ningún trianglo, ya que cuando no tiene ningún vértice marcado indica que esta completamente fuera de la superficie, o en el otro caso, el cubo está contenido completamente dentro de la superficie, y ambos casos deben ser reemplazados por un cubo con cero triangulos.

\section{Datos de entrada}
\label{ch:implementacion:sec:datosDeEntrada}

Para poder utilizar el algoritmo de \emph{MarchingCubes}, se necesita que cada punto de un espacio tridimensional tenga asociado un valor dentro de un rango definido de manera que cada punto pueda describirse como un vector de cuatro dimensiones compuestas por sus coordenadas y un valor escalar, como muestra la ecuacion \ref{ch:implementacion:sec:datosDeEntrada:vector}.

\begin{equation}
\label{ch:implementacion:sec:datosDeEntrada:vector}
	(x,y,z,v)
\end{equation}

Por ejemplo, en una habitación se tiene un foco de luz en el centro emitiendo luz en todas direcciones, con esto, cada punto de la habitación percibe distintas instensidades de luz que depende de la distancia al foco, mientras mas cerca del foco, mayor intensidad, mientras mas alejado del foco, menor intensidad. Si de alguna manera pudiesen marcase todos aquellos puntos que tengan una intensidad especifica se tendria una esfera formada por puntos equidistantes al foco, ya que todo ellos comparten la misma intensidad, por lo que se dice que esa es la superficie que encierra a todos los puntos con una cierta intensidad o mayor.

Ciertamente, el algoritmo tambien sirve para generar visualizar graficos en tres dimensiones, usando el mismo principio, cada punto del espacio tiene un valor asociado el cual es calculado por la ecuacion, para poder dibujar la susperficie se puede transformar la ecuacion en una inecuacion y luego marcar todos aquellos puntos que sean menores que el valor entregado por la ecuacion, por ejemplo se tiene en la ecuación \ref{ch:implementacion:sec:datosDeEntrada:ec1}:

\begin{equation}
\label{ch:implementacion:sec:datosDeEntrada:ec1}
	x^{2} + y^{2} = 1
\end{equation}

Para poder graficar esta ecuacion usando \emph{MarchingCubes} se debe transformar la ecuación en una inecuación como se muestra en la ecuacion \ref{ch:implementacion:sec:datosDeEntrada:ec2}:

\begin{equation}
\label{ch:implementacion:sec:datosDeEntrada:ec2}
	x^{2} + y^{2} <= 1
\end{equation}

Dependiendo de la resolución escogida, es posible que algunos cubos tengan algunos de sus vértices con coordenadas $(x,y)$ que satisfacen a la inecuación, por lo tanto quedan marcados como vértices internos, luego el proceso es ir iterando en todos esos cubos y reemplazando los cubos por los triangulos que correspondan a los quince casos mostrados en la figura \ref{f:estadoDelArte:MarchingCubes}.

Otra forma de entregar datos de entrada es teniendo cortes de nivel de un cuerpo tridimensional, un ejemplo de esto son las imágenes obtenidas de los exámenes de resonancia magnética. Cada imagen representa un corte transversal del cuerpo estudiado, como el que se muestra en la figura \ref{f:ch:implementacion:sec:datosdeentrada:img:ejemplodernm}:

\begin{figure}[hbt]
	\makebox[\textwidth]{\framebox[0.5\textwidth]{\rule{0pt}{0.5\textwidth}}}
	\caption{Imágenes una resonancia magnética de abdomen completo}
	\label{f:ch:implementacion:sec:datosdeentrada:img:ejemplodernm}
\end{figure}

Éstas imágenes, en conjunto, describen un cuerpo en tres dimensiones, mostrando con detalle sus componentes internos. Cada imágen está compuesta por una matriz de pixeles en una escala de grises, dependiendo de la profundidad de color, pueden ser de 256 valores (8 bits) o de 65536 valores (16 bits) diferentes, con estas imágenes se puede construir un espacio que puede servir como entrada para el algoritmo de \emph{MarchingCubes}, una posible manera es usando las coordenadas en 2 dimensiones del pixel en una imagen, y el valor z que representa la imagen dentro del cuerpo, luego el valor de este punto esta determinado por la intensidad de color del pixel, como se muestra en la ecuacion \ref{ch:implementacion:sec:datosDeEntrada:ec3}:

\begin{equation}
\label{ch:implementacion:sec:datosDeEntrada:ec3}
	(x,y,z,color)
\end{equation}

