\chapter{Estado del Arte}
\label{ch:estadoDelArte}

En el caso de las imágenes obtenidas por resonancia magnética, lo que se obtiene son un 
arreglo de imágenes de distintos niveles en un mismo eje geométrico de un cuerpo. Luego, para 
generar una malla geométrica del cuerpo existen diversas técnicas, muchas de ellas basadas en 
\jcq{voxels} como por ejemplo, Marching Cubes, Marching Tetrahedrons y Marching Diamonds.
En esta investigación se dará especial énfasis en el algoritmo de Marching Cubes, se 
explicará su funcionamiento, resultados y problemas que actualmente presenta.
Para entender mejor cómo funciona Marching Cubes, es de utilidad primero ver el caso 
reducido a dos dimensiones, el que puede ser llamado como \jcq{Marching Squares}.

\section{Marching Squares}
\label{sec:marchingSquares}

En dos dimensiones, para obtener el contorno de una figura, se puede usar este algoritmo 
que se basa en crear divisiones cuadradas uniformes del espacio, y en cada una de ellas dibujar un 
patrón específico del caso que encierra esa división, finalmente todas las divisiones tienen 
dibujadas un patrón específico o un espacio vacío, obteniendo así el contorno de la imagen 
tratada. El siguiente caso, ejemplifica el proceso completo.

Se supone la imagen de la figura \ref{f:estadoDelArte:original} para tratar obtener su contorno.

\begin{figure}
\centering
	\fbox{
		\includegraphics[width=0.7\textwidth]{images/marchingsquare/original.jpg}
	}
\caption{Figura original de estudio, a la que se desea extraer su contorno}
\label{f:estadoDelArte:original}
\end{figure}

Luego se divide el espacio en un arreglo uniforme de cuadriláteros, en este caso una 
matriz de 7x8 celdas como muestra la figura \ref{f:estadoDelArte:division}

Para poder obtener el patrón a dibujar en cada celda, se hace un proceso de 
reconocimiento. En un principio es necesario etiquetar cada vértice de cada celda dependiendo si 
están dentro o fuera de la región de interés. Esto puede ser calculado de distintas formas que 
dependen intrínsecamente del problema, para este caso, basta con detectar si el vértice está sobre 
un pixel blanco o no, en otros casos, puede depender de que si el valor del pixel en escala de 
grises supera (o es inferior) a un valor fijo determinado por el científico.

\begin{figure}
\centering
	\includegraphics[width=0.7\textwidth]{images/marchingsquare/division.jpg}
\caption{Figura original dividida en un arreglo uniforme de celdas}
\label{f:estadoDelArte:division}
\end{figure}

A continuación, se muestra como quedan etiquetados los vértices, dependiendo si están 
dentro de la figura (en rojo), o si están fuera de la figura (en azul), como muestra la figura \ref{f:estadoDelArte:labeledobj}

\begin{figure}
\centering
	\includegraphics[width=0.7\textwidth]{images/marchingsquare/labeledobj.jpg}
\caption{Etiquetado de vértices según su condición relativa a la figura}
\label{f:estadoDelArte:labeledobj}
\end{figure}

Si ahora se analiza celda a celda, se ve que existen dieciséis combinaciones distintas de 
vértices etiquetados como dentro o fuera de la región en una misma celda, como muestra la figura 
\ref{f:estadoDelArte:cases}

\begin{figure}
\centering
	\fbox{
		\includegraphics[width=\textwidth]{images/marchingsquare/cases.jpg}
	}
\caption{Los dieciséis casos posibles en Marching Squares}
\label{f:estadoDelArte:cases}
\end{figure}

Además se entiende que si uno de los lados de una celda (una arista) esta formada por un 
vértice marcado como dentro y otro fuera, significa que por esa arista corta el borde de la figura, 
y por lo tanto, esa arista la dejamos marcada como una arista de contorno detectada, que en este 
ejemplo se marca de color morado, como muestra la figura \ref{f:estadoDelArte:purpledobj}

\begin{figure}
\centering
	\includegraphics[width=0.7\textwidth]{images/marchingsquare/purpledobj.jpg}
\caption{Aristas marcadas por las cuales cortará el contorno calculado}
\label{f:estadoDelArte:purpledobj}
\end{figure}

Luego, por cada celda, se unen los puntos morados formados en el paso anterior, creando 
así uno de los dieciséis patrones posibles, lo cual se puede entender como una de las dieciséis 
formas en que un cuadrilátero puede ser atravesado por una linea, formando así el contorno 
buscado, como muestra la figura \ref{f:estadoDelArte:connectedobj}

\begin{figure}
\centering
	\includegraphics[width=0.7\textwidth]{images/marchingsquare/connectedobj.jpg}
\caption{Contorno calculado usando las aristas marcadas en el paso anterior}
\label{f:estadoDelArte:connectedobj}
\end{figure}

Posteriormente, dependiendo de los requerimientos de la investigación, se puede mejorar 
la aproximación haciendo una interpolación lineal de los valores de los vértices para calcular en 
que punto aproximadamente la figura (a la que se quiere extraer su contorno), corta con la arista, 
un ejemplo del resultado de esta aproximación se muestra en la figura \ref{f:estadoDelArte:2Dintersected}

\begin{figure}
\centering
	\includegraphics[width=0.7\textwidth]{images/marchingsquare/2Dintersected.jpg}
\caption{Mejorando la calidad del contorno usando interpolación lineal}
\label{f:estadoDelArte:2Dintersected}
\end{figure}

\subsection{Consecuencias}
\label{subsec:consecuencias}

Evidentemente, existen ciertas falencias, por ejemplo, el contorno obtenido no simula 
adecuadamente el objeto estudiado, debido a errores causados por la fragmentación de las 
divisiones iniciales, una solución directa para mejorar esto es aumentar las divisiones, es decir, 
hacer que todas las celdas sean mas pequeñas, y así hacer una mejor aproximación del contorno 
del objeto en estudio. De la misma manera, existen otras técnicas, tales como hacer particiones 
con celdas de tamaño variable en las particiones, o subdividir aquellas celdas que hayan sido 
detectadas como de frontera y así obtener un mejor desempeño en el algoritmo.

Otro problema es que algunos casos presentan ambigüedad, es decir, no es trivial calcular 
a cual caso pertenece una cierta configuración, por ejemplo, tomando el quinto y décimo caso 
descritos anteriormente, se supone el ejemplo descrito por la figura \ref{f:estadoDelArte:marchingSAmbEx}

\begin{figure}[hbp]
\centering
	\fbox{
		\includegraphics[width=0.7\textwidth]{images/marchingsquare/marchingSAmbEx.jpg}
	}
\caption{Casos con ambigüedad}
\label{f:estadoDelArte:marchingSAmbEx}
\end{figure}

Este cuadrado, tiene dos vértices diagonalmente opuestos marcados. Sin conocer como es 
la figura ni cómo son las divisiones vecinas, no se puede saber con exactitud si se trata del quinto 
o el décimo caso, por lo que el algoritmo puede erróneamente separar el contorno, formando así 
dos figuras separadas, o las une, de manera que sólo exista una figura con un contorno 
compartido.

\section{Marching Cubes}
\label{sec:marchingCubes}

\subsection{Idea}
\label{subsec:idea}

Marching Cubes es un algoritmo de extracción de una superficie poligonal de un cuerpo 
en un espacio escalar en tres dimensiones. Existen muchas aplicaciones para este tipo de técnicas, 
dos de las más comunes son:

\begin{itemize}
	\item Reconstrucción de una superficie a partir de un set de imágenes médicas, como 
	por ejemplo los obtenidos en imágenes de resonancia magnética, los que pueden formar 
	un volumen en tres dimensiones.

	\item Crear un contorno tridimensional de un campo escalar matemático, en este caso, 
	el valor de una cierta función es conocido en todo el espacio, pero es representada como 
	vértices de una malla tridimensional.
\end{itemize}

Adopta la misma idea que hay detrás de Marching Squares, pero llevando los conceptos a 
tres dimensiones, en este caso, el dominio es un espacio tridimensional, en el cual existe un 
cuerpo al que se desea extraer su superficie. Luego, el espacio es dividido en regiones uniformes 
(cubos), por los cuales la superficie del objeto corta las aristas de estos cubos.

\subsection{Consideraciones Geométricas}
\label{subsec:consideracionesGeometricas}

Un cubo tiene seis caras, ocho vértices y doce aristas, las cuales, para efectos de esta 
investigación serán numeradas de la siguiente manera:

