\chapter{Estado del Arte}
\label{ch:estadoDelArte}

En el caso de las imágenes obtenidas por resonancia magnética, lo que se obtiene son un 
arreglo de imágenes de distintos niveles en un mismo eje geométrico de un cuerpo. Luego, para 
generar una malla geométrica del cuerpo existen diversas técnicas, muchas de ellas basadas en 
\jcq{voxels} como por ejemplo, Marching Cubes, Marching Tetrahedrons y Marching Diamonds.
En esta investigación se dará especial énfasis en el algoritmo de Marching Cubes, se 
explicará su funcionamiento, resultados y problemas que actualmente presenta.
Para entender mejor cómo funciona Marching Cubes, es de utilidad primero ver el caso 
reducido a dos dimensiones, el que puede ser llamado como \jcq{Marching Squares}.

\section{Marching Squares}
\label{sec:marchingSquares}

En dos dimensiones, para obtener el contorno de una figura, se puede usar este algoritmo 
que se basa en crear divisiones cuadradas uniformes del espacio, y en cada una de ellas dibujar un 
patrón específico del caso que encierra esa división, finalmente todas las divisiones tienen 
dibujadas un patrón específico o un espacio vacío, obteniendo así el contorno de la imagen 
tratada. El siguiente caso, ejemplifica el proceso completo.

Se supone la imagen de la figura \ref{f:estadoDelArte:original} para tratar obtener su contorno.

\begin{figure}
\centering
	\fbox{
		\includegraphics[width=0.7\textwidth]{images/marchingsquare/original.jpg}
	}
\caption{Figura original de estudio, a la que se desea extraer su contorno}
\label{f:estadoDelArte:original}
\end{figure}

Luego se divide el espacio en un arreglo uniforme de cuadriláteros, en este caso una 
matriz de 7x8 celdas como muestra la figura \ref{f:estadoDelArte:division}

Para poder obtener el patrón a dibujar en cada celda, se hace un proceso de 
reconocimiento. En un principio es necesario etiquetar cada vértice de cada celda dependiendo si 
están dentro o fuera de la región de interés. Esto puede ser calculado de distintas formas que 
dependen intrínsecamente del problema, para este caso, basta con detectar si el vértice está sobre 
un pixel blanco o no, en otros casos, puede depender de que si el valor del pixel en escala de 
grises supera (o es inferior) a un valor fijo determinado por el científico.

\begin{figure}
\centering
	\includegraphics[width=0.7\textwidth]{images/marchingsquare/division.jpg}
\caption{Figura original dividida en un arreglo uniforme de celdas}
\label{f:estadoDelArte:division}
\end{figure}

A continuación, se muestra como quedan etiquetados los vértices, dependiendo si están 
dentro de la figura (en rojo), o si están fuera de la figura (en azul), como muestra la figura \ref{f:estadoDelArte:labeledobj}

\begin{figure}
\centering
	\includegraphics[width=0.7\textwidth]{images/marchingsquare/labeledobj.jpg}
\caption{Etiquetado de vértices según su condición relativa a la figura}
\label{f:estadoDelArte:labeledobj}
\end{figure}

Si ahora se analiza celda a celda, se ve que existen dieciséis combinaciones distintas de 
vértices etiquetados como dentro o fuera de la región en una misma celda, como muestra la figura 
\ref{f:estadoDelArte:cases}

\begin{figure}
\centering
	\fbox{
		\includegraphics[width=\textwidth]{images/marchingsquare/cases.jpg}
	}
\caption{Los dieciséis casos posibles en Marching Squares}
\label{f:estadoDelArte:cases}
\end{figure}

Además se entiende que si uno de los lados de una celda (una arista) esta formada por un 
vértice marcado como dentro y otro fuera, significa que por esa arista corta el borde de la figura, 
y por lo tanto, esa arista la dejamos marcada como una arista de contorno detectada, que en este 
ejemplo se marca de color morado, como muestra la figura \ref{f:estadoDelArte:purpledobj}

\begin{figure}
\centering
	\includegraphics[width=0.7\textwidth]{images/marchingsquare/purpledobj.jpg}
\caption{Aristas marcadas por las cuales cortará el contorno calculado}
\label{f:estadoDelArte:purpledobj}
\end{figure}

Luego, por cada celda, se unen los puntos morados formados en el paso anterior, creando 
así uno de los dieciséis patrones posibles, lo cual se puede entender como una de las dieciséis 
formas en que un cuadrilátero puede ser atravesado por una linea, formando así el contorno 
buscado, como muestra la figura \ref{f:estadoDelArte:connectedobj}

\begin{figure}
\centering
	\includegraphics[width=0.7\textwidth]{images/marchingsquare/connectedobj.jpg}
\caption{Contorno calculado usando las aristas marcadas en el paso anterior}
\label{f:estadoDelArte:connectedobj}
\end{figure}

Posteriormente, dependiendo de los requerimientos de la investigación, se puede mejorar 
la aproximación haciendo una interpolación lineal de los valores de los vértices para calcular en 
que punto aproximadamente la figura (a la que se quiere extraer su contorno), corta con la arista, 
un ejemplo del resultado de esta aproximación se muestra en la figura \ref{f:estadoDelArte:2Dintersected}

\begin{figure}
\centering
	\includegraphics[width=0.7\textwidth]{images/marchingsquare/2Dintersected.jpg}
\caption{Mejorando la calidad del contorno usando interpolación lineal}
\label{f:estadoDelArte:2Dintersected}
\end{figure}

\subsection{Consecuencias}
\label{subsec:marchingSquares:consecuencias}

Evidentemente, existen ciertas falencias, por ejemplo, el contorno obtenido no simula 
adecuadamente el objeto estudiado, debido a errores causados por la fragmentación de las 
divisiones iniciales, una solución directa para mejorar esto es aumentar las divisiones, es decir, 
hacer que todas las celdas sean mas pequeñas, y así hacer una mejor aproximación del contorno 
del objeto en estudio. De la misma manera, existen otras técnicas, tales como hacer particiones 
con celdas de tamaño variable en las particiones, o subdividir aquellas celdas que hayan sido 
detectadas como de frontera y así obtener un mejor desempeño en el algoritmo.

Otro problema es que algunos casos presentan ambigüedad, es decir, no es trivial calcular 
a cual caso pertenece una cierta configuración, por ejemplo, tomando el quinto y décimo caso 
descritos anteriormente, se supone el ejemplo descrito por la figura \ref{f:estadoDelArte:marchingSAmbEx}

\begin{figure}[hbp]
\centering
	\fbox{
		\includegraphics[width=0.7\textwidth]{images/marchingsquare/marchingSAmbEx.jpg}
	}
\caption{Casos con ambigüedad}
\label{f:estadoDelArte:marchingSAmbEx}
\end{figure}

Este cuadrado, tiene dos vértices diagonalmente opuestos marcados. Sin conocer como es 
la figura ni cómo son las divisiones vecinas, no se puede saber con exactitud si se trata del quinto 
o el décimo caso, por lo que el algoritmo puede erróneamente separar el contorno, formando así 
dos figuras separadas, o las une, de manera que sólo exista una figura con un contorno 
compartido.

\section{Marching Cubes}
\label{sec:marchingCubes}

\subsection{Idea}
\label{subsec:marchingCubes:idea}

Marching Cubes es un algoritmo de extracción de una superficie poligonal de un cuerpo 
en un espacio escalar en tres dimensiones. Existen muchas aplicaciones para este tipo de técnicas, 
dos de las más comunes son:

\begin{itemize}
	\item Reconstrucción de una superficie a partir de un set de imágenes médicas, como 
	por ejemplo los obtenidos en imágenes de resonancia magnética, los que pueden formar 
	un volumen en tres dimensiones.

	\item Crear un contorno tridimensional de un campo escalar matemático, en este caso, 
	el valor de una cierta función es conocido en todo el espacio, pero es representada como 
	vértices de una malla tridimensional.
\end{itemize}

Adopta la misma idea que hay detrás de Marching Squares, pero llevando los conceptos a 
tres dimensiones, en este caso, el dominio es un espacio tridimensional, en el cual existe un 
cuerpo al que se desea extraer su superficie. Luego, el espacio es dividido en regiones uniformes 
(cubos), por los cuales la superficie del objeto corta las aristas de estos cubos.

\subsection{Consideraciones Geométricas}
\label{subsec:marchingCubes:consideracionesGeometricas}

Un cubo tiene seis caras, ocho vértices y doce aristas, las cuales, para efectos de esta 
investigación serán numeradas como se muestra en la figura \ref{f:estadoDelArte:convention}

\begin{figure}[hbp]
\centering
	\fbox{
		\includegraphics[width=0.9\textwidth]{images/marchingcubes/convention.jpg}
	}
\caption{Convención de enumeración de vértices y aristas}
\label{f:estadoDelArte:convention}
\end{figure}

De la misma manera que el caso de dos dimensiones, debido a que cada uno de los ocho 
vértices puede tener dos estados: vértice marcado como interno o externo (dentro o fuera de la 
superficie del cuerpo), se tienen 256 combinaciones posibles, es decir, una superficie puede
atravesar a un cubo de 256 maneras posibles. Sin embargo, al igual que en su homólogo en dos 
dimensiones, que tiene dieciséis formas posibles, estas pueden ser reducidas a un numero inferior 
de patrones, ya que entre ellos existen diferencias solamente de rotación y reflexión.

En el caso de tres dimensiones, los 256 patrones pueden ser reducidos de la misma 
manera. Dos de esos casos son triviales, ya que tienen todos sus vértices marcados como internos 
o externos, por lo tanto, ambos casos no contribuyen a la superficie que se quiere extraer.

Si se consideran las simetrías, hay solamente catorce posibles configuraciones únicas en 
los restantes 254 casos. Por ejemplo, aquellos casos que solamente tienen un vértice marcado 
como interno, sólo representan un triángulo que atraviesa las aristas que convergen a ése vértice y 
existen ocho casos como éste, uno por cada vértice del cubo.

Finalmente las quince familias de posibles casos son los que se describen en la figura \ref{f:estadoDelArte:MarchingCubes}

\begin{figure}[hbp]
\centering
	\fbox{
		\includegraphics[width=0.9\textwidth]{images/marchingcubes/MarchingCubes.pdf}
	}
\caption{Los quince casos de Marching Cubes}
\label{f:estadoDelArte:MarchingCubes}
\end{figure}

\subsection{Procedimiento}
\label{subsec:marchingCubes:procedimiento}

El procedimiento es similar al explicado a Marching Squares, en primer lugar se divide el 
espacio en un arreglo uniforme de regiones cúbicas, luego, se evalúa cada vértice para etiquetarlo 
como un vértice interno o externo, luego, calcular a cuál de los 15 casos pertenece cada división, 
y generar los triángulos, para que finalmente, extraer así la superficie.
En resumen, el procedimiento reducido a un solo cubo, ocurre según lo siguiente:

En un comienzo se tiene un cubo, como el descrito en la figura \ref{f:estadoDelArte:cube_01}

\begin{figure}[hbp]
\centering
	\fbox{
		\includegraphics[width=0.4\textwidth]{images/marchingcubes/cube_01.png}
	}
\caption{Un cubo con sus vértices marcados}
\label{f:estadoDelArte:cube_01}
\end{figure}

Suponiendo que este cubo tiene solamente un vértice que quedó dentro del cuerpo, éste 
vértice queda marcado como un vértice interno (en rojo), los siete restantes quedan marcados 
como vértices externos (en negro), descrito por le figura \ref{f:estadoDelArte:cube_02}

\begin{figure}[hbp]
\centering
	\fbox{
		\includegraphics[width=0.4\textwidth]{images/marchingcubes/cube_02.png}
	}
\caption{Un cubo con uno de sus vértices marcado como interno}
\label{f:estadoDelArte:cube_02}
\end{figure}

Luego se crea un triángulo cuyos vértices se apoyan en los puntos medios de las aristas 
que comparten el vértice marcado como interno, de esta manera, se tiene un cubo que es 
atravesado por una superficie que precisamente deja un sólo vértice dentro (o fuera, dependiendo 
de la reflexión del caso), resultando un triangulo como el de la figura \ref{f:estadoDelArte:cube_03}

\begin{figure}[ht]
\centering
	\fbox{
		\includegraphics[width=0.4\textwidth]{images/marchingcubes/cube_03.png}
	}
\caption{Un triángulo atraviesa el cubo separando el vértice marcado de los demás}
\label{f:estadoDelArte:cube_03}
\end{figure}

\subsection{Consecuencias}
\label{subsec:marchingCubes:consecuencias}

Nuevamente, al igual que su versión en dos dimensiones, presenta las mismas falencias, 
por ejemplo, sin ninguna interpolación, y dependiendo de la resolución de la división, la 
superficie extraída puede presentar un efecto de escalonamiento (\jcq{aliasing}) como el que se 
muestra en la figura \ref{f:estadoDelArte;superficies_resultantes_variando_divisiones}

\begin{figure}

	\begin{subfigure}{0.45\textwidth}
		\centering
		\includegraphics[width=\textwidth]{images/marchingcubes/mc_blobs.png}
		\caption{La forma original}
		\label{f:estadoDelArte:mc_blobs}
	\end{subfigure}
	~
	\begin{subfigure}{0.45\textwidth}
		\centering
		\includegraphics[width=\textwidth]{images/marchingcubes/mc_blobs_12.png}
		\caption{Marching Cubes con 12 divisiones}
		\label{f:estadoDelArte:mc_blobs_12}
	\end{subfigure}

	\begin{subfigure}{0.45\textwidth}
		\centering
		\includegraphics[width=\textwidth]{images/marchingcubes/mc_blobs_50.png}
		\caption{Marching Cubes con 50 divisiones}
		\label{f:estadoDelArte:mc_blobs_50}
	\end{subfigure}
	~
	\begin{subfigure}{0.45\textwidth}
		\centering
		\includegraphics[width=\textwidth]{images/marchingcubes/mc_blobs_20.png}
		\caption{Marching Cubes con 20 divisiones}
		\label{f:estadoDelArte:mc_blobs_20}
	\end{subfigure}

	\caption{Superficies resultantes variando la cantidad de divisiones}
	\label{f:estadoDelArte;superficies_resultantes_variando_divisiones}
\end{figure}

Se puede apreciar que al aumentar la resolución, la calidad aumenta, pero 
computacionalmente requiere mas cómputo y memoria, ya que aumentan la cantidad de caras que 
describen la superficie extraída, como se ve en la figura \ref{f:estadoDelArte:polygonise3}

\begin{figure}[hbp]
\centering
	\fbox{
		\includegraphics[width=0.9\textwidth]{images/marchingcubes/polygonise3.png}
	}
\caption{Ejemplificación de cómo varían los resultados al modificar la resolución}
\label{f:estadoDelArte:polygonise3}
\end{figure}

Otro problema importante son los posibles problemas topológicos que tiene este 
algoritmo, debido a los casos ambiguos que se explican de la misma manera que en su versión en 
dos dimensiones, y algunos casos donde se forman agujeros dependiendo de que ciertos casos 
queden adyacentes.

Por ejemplo, si se juntan un caso especial de dos vértices opuestos marcados como 
internos que comparten la misma cara y otro caso especial en el que cinco vértices están 
marcados como internos, pueden generar agujeros en el modelo final en tres dimensiones, como 
se gráfica en la figura \ref{f:estadoDelArte:MCAmbEx}

\begin{figure}[htbp]
\centering
	\fbox{
		\includegraphics[width=0.9\textwidth]{images/marchingcubes/MCAmbEx.png}
	}
\caption{Casos en los que se generan errores topológicos}
\label{f:estadoDelArte:MCAmbEx}
\end{figure}

Para poder evitar estos errores, se introducen seis nuevas familias de casos, que están descritas en
la figura \ref{f:estadoDelArte:MCAmb}

\begin{figure}[htb]
\centering
	\fbox{
		\includegraphics[width=0.8\textwidth]{images/marchingcubes/MCAmb.png}
	}
\caption{Nuevos patrones que solucionan los errores topológicos}
\label{f:estadoDelArte:MCAmb}
\end{figure}

En el ejemplo anterior, en vez de usar el patrón etiquetado como 6e, debería usarse el 6c 
de la figura anterior, solucionando así el potencial error topológico.
Otra posible solución, es usar pirámides o tetraedros (Marching Tetrahedrons), los cuales 
son más sencillas de utilizar ya que sólo existen 16 posibles combinaciones totales (al igual que 
Marching Squares, ya que en ambos casos, los elementos tienen cuatro vértices), los cuales 
pueden ser reducidos a sólo cuatro. La poligonización tetraédrica se muestra en la figura \ref{f:estadoDelArte:image_004}

\begin{figure}[htb]
\centering
	\fbox{
		\includegraphics[width=0.9\textwidth]{images/marchingtetrahedrons/image_004.png}
	}
\caption{Poligonización tetraédrica}
\label{f:estadoDelArte:image_004}
\end{figure}

De hecho, en el caso de Marching Cubes, cada cubo puede al final, descomponerse en 
cinco pirámides o tetraedros como se muestra en la figura \ref{f:estadoDelArte:image_006}

\begin{figure}[htb]
\centering
	\fbox{
		\includegraphics[width=0.7\textwidth]{images/marchingtetrahedrons/image_006.png}
	}
\caption{Descomposición tetraédrica de un cubo}
\label{f:estadoDelArte:image_006}
\end{figure}
