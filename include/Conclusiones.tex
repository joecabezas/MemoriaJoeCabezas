\chapter{Conclusiones}
\label{ch:conclusiones}

Las mallas geométricas son una gran herramienta dentro de la ciencia e ingeniería, sus 
aplicaciones son muchas, y en general se usan para representar un objeto o cuerpo 
discretizadamente.

Por esto, y debido a las implicaciones físicas, es que no se puede hacer un estudio 
infinitesimal de los cuerpos en tres dimensiones y es necesario crear modelos discretizados de 
ellos, para facilitar el trabajo con estos.

Para los fines de visualización, existen diversas técnicas basadas en elementos finitos 
como Marching Squares, Marching Cubes y Marching Tetrahedrons, los cuales, dependiendo de 
su dimensión y objetivos, son capaces de extraer de manera discretizada el contorno o superficie 
de un cuerpo en estudio, para finalmente poder trabajar con él.

No obstante, estos métodos presentan problemas causados por la discretización misma que 
genera malas aproximaciones y para mejorarlas requieren de mayor poder de cómputo, o también 
presentan problemas topológicos causando errores de cálculo importantes. Aún así, hasta la fecha 
se han investigado y aún se pueden crear nuevas técnicas para poder enfrentar estas situaciones, y 
generar mallas cada vez de mayor calidad.
